\documentclass[11pt,]{problemset}
\usepackage{lmodern}
\usepackage{amssymb,amsmath}
\usepackage{ifxetex,ifluatex}
\usepackage{fixltx2e} % provides \textsubscript
\ifnum 0\ifxetex 1\fi\ifluatex 1\fi=0 % if pdftex
  \usepackage[T1]{fontenc}
  \usepackage[utf8]{inputenc}
\else % if luatex or xelatex
  \ifxetex
    \usepackage{mathspec}
  \else
    \usepackage{fontspec}
  \fi
  \defaultfontfeatures{Ligatures=TeX,Scale=MatchLowercase}
\fi
% use upquote if available, for straight quotes in verbatim environments
\IfFileExists{upquote.sty}{\usepackage{upquote}}{}
% use microtype if available
\IfFileExists{microtype.sty}{%
\usepackage{microtype}
\UseMicrotypeSet[protrusion]{basicmath} % disable protrusion for tt fonts
}{}
\usepackage{hyperref}
\hypersetup{unicode=true,
            pdfborder={0 0 0},
            breaklinks=true}
\urlstyle{same}  % don't use monospace font for urls
\usepackage{color}
\usepackage{fancyvrb}
\newcommand{\VerbBar}{|}
\newcommand{\VERB}{\Verb[commandchars=\\\{\}]}
\DefineVerbatimEnvironment{Highlighting}{Verbatim}{commandchars=\\\{\}}
% Add ',fontsize=\small' for more characters per line
\usepackage{framed}
\definecolor{shadecolor}{RGB}{248,248,248}
\newenvironment{Shaded}{\begin{snugshade}}{\end{snugshade}}
\newcommand{\AlertTok}[1]{\textcolor[rgb]{0.94,0.16,0.16}{#1}}
\newcommand{\AnnotationTok}[1]{\textcolor[rgb]{0.56,0.35,0.01}{\textbf{\textit{#1}}}}
\newcommand{\AttributeTok}[1]{\textcolor[rgb]{0.77,0.63,0.00}{#1}}
\newcommand{\BaseNTok}[1]{\textcolor[rgb]{0.00,0.00,0.81}{#1}}
\newcommand{\BuiltInTok}[1]{#1}
\newcommand{\CharTok}[1]{\textcolor[rgb]{0.31,0.60,0.02}{#1}}
\newcommand{\CommentTok}[1]{\textcolor[rgb]{0.56,0.35,0.01}{\textit{#1}}}
\newcommand{\CommentVarTok}[1]{\textcolor[rgb]{0.56,0.35,0.01}{\textbf{\textit{#1}}}}
\newcommand{\ConstantTok}[1]{\textcolor[rgb]{0.00,0.00,0.00}{#1}}
\newcommand{\ControlFlowTok}[1]{\textcolor[rgb]{0.13,0.29,0.53}{\textbf{#1}}}
\newcommand{\DataTypeTok}[1]{\textcolor[rgb]{0.13,0.29,0.53}{#1}}
\newcommand{\DecValTok}[1]{\textcolor[rgb]{0.00,0.00,0.81}{#1}}
\newcommand{\DocumentationTok}[1]{\textcolor[rgb]{0.56,0.35,0.01}{\textbf{\textit{#1}}}}
\newcommand{\ErrorTok}[1]{\textcolor[rgb]{0.64,0.00,0.00}{\textbf{#1}}}
\newcommand{\ExtensionTok}[1]{#1}
\newcommand{\FloatTok}[1]{\textcolor[rgb]{0.00,0.00,0.81}{#1}}
\newcommand{\FunctionTok}[1]{\textcolor[rgb]{0.00,0.00,0.00}{#1}}
\newcommand{\ImportTok}[1]{#1}
\newcommand{\InformationTok}[1]{\textcolor[rgb]{0.56,0.35,0.01}{\textbf{\textit{#1}}}}
\newcommand{\KeywordTok}[1]{\textcolor[rgb]{0.13,0.29,0.53}{\textbf{#1}}}
\newcommand{\NormalTok}[1]{#1}
\newcommand{\OperatorTok}[1]{\textcolor[rgb]{0.81,0.36,0.00}{\textbf{#1}}}
\newcommand{\OtherTok}[1]{\textcolor[rgb]{0.56,0.35,0.01}{#1}}
\newcommand{\PreprocessorTok}[1]{\textcolor[rgb]{0.56,0.35,0.01}{\textit{#1}}}
\newcommand{\RegionMarkerTok}[1]{#1}
\newcommand{\SpecialCharTok}[1]{\textcolor[rgb]{0.00,0.00,0.00}{#1}}
\newcommand{\SpecialStringTok}[1]{\textcolor[rgb]{0.31,0.60,0.02}{#1}}
\newcommand{\StringTok}[1]{\textcolor[rgb]{0.31,0.60,0.02}{#1}}
\newcommand{\VariableTok}[1]{\textcolor[rgb]{0.00,0.00,0.00}{#1}}
\newcommand{\VerbatimStringTok}[1]{\textcolor[rgb]{0.31,0.60,0.02}{#1}}
\newcommand{\WarningTok}[1]{\textcolor[rgb]{0.56,0.35,0.01}{\textbf{\textit{#1}}}}
\usepackage{graphicx,grffile}
\makeatletter
\def\maxwidth{\ifdim\Gin@nat@width>\linewidth\linewidth\else\Gin@nat@width\fi}
\def\maxheight{\ifdim\Gin@nat@height>\textheight\textheight\else\Gin@nat@height\fi}
\makeatother
% Scale images if necessary, so that they will not overflow the page
% margins by default, and it is still possible to overwrite the defaults
% using explicit options in \includegraphics[width, height, ...]{}
\setkeys{Gin}{width=\maxwidth,height=\maxheight,keepaspectratio}
\IfFileExists{parskip.sty}{%
\usepackage{parskip}
}{% else
\setlength{\parindent}{0pt}
\setlength{\parskip}{6pt plus 2pt minus 1pt}
}
\setlength{\emergencystretch}{3em}  % prevent overfull lines
\providecommand{\tightlist}{%
  \setlength{\itemsep}{0pt}\setlength{\parskip}{0pt}}
\setcounter{secnumdepth}{5}
% Redefines (sub)paragraphs to behave more like sections
\ifx\paragraph\undefined\else
\let\oldparagraph\paragraph
\renewcommand{\paragraph}[1]{\oldparagraph{#1}\mbox{}}
\fi
\ifx\subparagraph\undefined\else
\let\oldsubparagraph\subparagraph
\renewcommand{\subparagraph}[1]{\oldsubparagraph{#1}\mbox{}}
\fi

%%% Use protect on footnotes to avoid problems with footnotes in titles
\let\rmarkdownfootnote\footnote%
\def\footnote{\protect\rmarkdownfootnote}

%%% Change title format to be more compact
\usepackage{titling}

% Create subtitle command for use in maketitle
\providecommand{\subtitle}[1]{
  \posttitle{
    \begin{center}\large#1\end{center}
    }
}

\setlength{\droptitle}{-2em}

  \title{}
    \pretitle{\vspace{\droptitle}}
  \posttitle{}
    \author{}
    \preauthor{}\postauthor{}
    \date{}
    \predate{}\postdate{}
  
\usepackage[ipaex]{zxjafont}
\usepackage{sectsty}
\usepackage{tikz}
\usetikzlibrary{trees}
\usepackage{mathspec}
\usepackage{amsmath,amsthm}
\usepackage{amssymb,amsfonts}
\usepackage[all,arc]{xy}
\usepackage{enumerate}
\usepackage{mathrsfs}
\usepackage[margin=1in]{geometry}
\usepackage{thmtools}
\usepackage{verbatim}
\usepackage{xltxtra}
\XeTeXlinebreaklocale ``ja''
\XeTeXlinebreakskip=0pt plus 1pt
\XeTeXlinebreakpenalty=0
\usepackage{afterpage}
\sectionfont{\fontsize{12}{15}\selectfont}
\author{学籍番号:}
\course{疫学演習 (2019) 回答用紙}
\problemset{ }
\problem{ }
\collab{氏名:}

\begin{document}

\hypertarget{-20}{%
\section{問題1:両群間計量データの平均値を比較する (20\%)}\label{-20}}

\begin{enumerate}
\def\labelenumi{\arabic{enumi}.}
\tightlist
\item
  帰無仮説を「遺伝子変異ありと変異なし両群の間で,COGの平均値は等しい」とする.上記のデータ及び適切な方法を使って検定し,検定の結果を分かりやすく説明せよ.なお,分散が等しいと仮定できる場合,以下の式で両群の共通標準偏差が計算できる:(6\%)
\end{enumerate}

\begin{equation}
  \label{eq:1}
S = \sqrt{\frac{(n_A - 1)S^2_A + (n_B - 1)S^2_B}{n_A + n_B -2}}
\end{equation}

\begin{itemize}
\tightlist
\item
  \(S_A:\) A群の標準偏差;
\item
  \(n_A:\) A群の人数;
\item
  \(S_B:\) B群の標準偏差;
\item
  \(n_B:\) B群の人数;
\item
  \(S:\) A群及びB群の共通標準偏差;
\item
  \(n_A + n_B -2:\) 分散が等しい時の自由度.
\end{itemize}

\newpage 
\vfill

\begin{enumerate}
\def\labelenumi{\arabic{enumi}.}
\setcounter{enumi}{1}
\tightlist
\item
  遺伝子変異ありとなしの群の間の脳萎縮度 (atrophy) を比較する場合, 1.
  と同じ検定方法を用いてよいか?
  それを判断するにはどの検定方法を使えばよいかを説明し,実際にこの検定方法を実施せよ.(6\%)
\end{enumerate}

\newpage
\vfill

\begin{enumerate}
\def\labelenumi{\arabic{enumi}.}
\setcounter{enumi}{2}
\tightlist
\item
  2.の結果を踏まえて,帰無仮説「両群の脳萎縮度の平均値が等しい」を検定せよ.なお,両群の分散が等しいという前提が満たされていない時に,自由度(df)の計算式は以下となる:(8\%)
\end{enumerate}

\begin{equation}
\label{eq:2}
\mathbf{df} = \frac{(S^2_A/n_A + S^2_B/n_B)^2}{(S_A^2/n_A)^2/(n_A-1)+(S_B^2/n_B)^2/(n_B-1)}
\end{equation}

\newpage

\hypertarget{-30}{%
\section{問題2:線形回帰モデル (30\%)}\label{-30}}

2.3
年齢,体重それぞれの平均値,分散を求めよ;また,年齢と体重の共分散を算出せよ.なお,EZRで計量データの平均値を計算するには,コマンド
\texttt{mean(変数名)} を使う;共分散を計算したい時には,コマンド
\texttt{cov(変数1,\ 変数2)} を利用する.

以下のコードをRスクリプトに入力して,実行をクリックしてください.\newline
\underline{(結果を下の余白に記入すること)} (5\%)

\begin{Shaded}
\begin{Highlighting}[]
\CommentTok{# 年齢の平均値}
\KeywordTok{mean}\NormalTok{(Dataset}\OperatorTok{$}\NormalTok{age)}
\CommentTok{# 年齢の分散}
\KeywordTok{var}\NormalTok{(Dataset}\OperatorTok{$}\NormalTok{age)}
\CommentTok{# 体重の平均値}
\KeywordTok{mean}\NormalTok{(Dataset}\OperatorTok{$}\NormalTok{wt)}
\CommentTok{# 体重の分散}
\KeywordTok{var}\NormalTok{(Dataset}\OperatorTok{$}\NormalTok{wt)}
\CommentTok{# 体重と年齢の共分散 covariance}
\KeywordTok{cov}\NormalTok{(Dataset}\OperatorTok{$}\NormalTok{wt, Dataset}\OperatorTok{$}\NormalTok{age)}
\end{Highlighting}
\end{Shaded}

\newpage

2.4
年齢を説明変数,体重を目的変数とする場合,年齢の傾き(回帰係数),と切片を求めよ.なお,分散と共分散の定義は以下とする,\(\bar{X}\)
は \(X\) の平均値を示す:

\begin{itemize}
\tightlist
\item
  分散 variance:
\end{itemize}

\[
\begin{aligned}
\mathbf{Var}(X) & = \frac{(X_1-\bar{X})^2+(X_2-\bar{X})^2+\cdots+(X_n-\bar{X})^2}{n - 1} \\
                & = \frac{\sum_{i=1}^n(X_i-\bar{X})^2}{n -1}
\end{aligned}
\]

\begin{itemize}
\tightlist
\item
  共分散 covariance:
\end{itemize}

\[
\begin{aligned}
\mathbf{Cov}(X, Y) & = \frac{(X_1 - \bar{X})(Y_1-\bar{Y}) + (X_2 - \bar{X})(Y_2-\bar{Y}) + \cdots + (X_n - \bar{X})(Y_n-\bar{Y})}{n - 1} \\
                   & = \frac{\sum_{i = 1}^n(X_i - \bar{X})(Y_i-\bar{Y})}{n - 1}
\end{aligned}
\]

以下のコードをRスクリプトに入力して,実行をクリックしてください.\newline
\underline{(結果を下の余白に記入すること)} (2\%)

\begin{Shaded}
\begin{Highlighting}[]
\CommentTok{# 傾き (slope)}
\NormalTok{beta <-}\StringTok{ }\KeywordTok{cov}\NormalTok{(Dataset}\OperatorTok{$}\NormalTok{wt, Dataset}\OperatorTok{$}\NormalTok{age) }\OperatorTok{/}\StringTok{ }\KeywordTok{var}\NormalTok{(Dataset}\OperatorTok{$}\NormalTok{age)}
\NormalTok{beta}
\CommentTok{# 切片 (intercept)}
\NormalTok{alpha <-}\StringTok{ }\KeywordTok{mean}\NormalTok{(Dataset}\OperatorTok{$}\NormalTok{wt) }\OperatorTok{-}\StringTok{ }\KeywordTok{mean}\NormalTok{(Dataset}\OperatorTok{$}\NormalTok{age)}\OperatorTok{*}\NormalTok{beta}
\NormalTok{alpha}
\end{Highlighting}
\end{Shaded}

\newpage

2.6
今まで計算した傾きと切片の数字を用いて,年齢と体重の関係を線形と考える場合の計算式を記入せよ.傾きと切片の計算結果の意味をそれぞれ記述せよ.(4\%)

\bigskip\bigskip\bigskip\bigskip\bigskip
\bigskip\bigskip\bigskip\bigskip\bigskip
\bigskip\bigskip\bigskip\bigskip\bigskip

2.8
重回帰線形モデルの計算結果を用いて,体重の平均値を年齢と性別の線形モデルで表示せよ.各回帰係数の意味を説明せよ.(14\%)

\bigskip\bigskip\bigskip\bigskip\bigskip
\bigskip\bigskip\bigskip\bigskip\bigskip
\bigskip\bigskip\bigskip\bigskip\bigskip
\bigskip\bigskip\bigskip\bigskip\bigskip

2.9
上記の重回帰線形モデルを用いて,年齢が34ヶ月の女の子の体重の予測値を計算せよ.(5\%)

\newpage

\hypertarget{chi2--40}{%
\section{\texorpdfstring{問題3:\(\chi^2\)
検定,オッズ比,ロジスティック回帰モデル
(40\%)}{問題3:\textbackslash{}chi\^{}2 検定,オッズ比,ロジスティック回帰モデル (40\%)}}\label{chi2--40}}

3.1
もし,視覚障害と対象者の死亡リスクに関連がない場合,下の表(各セルの期待値の人数)を答えよ:(4\%)
\bigskip\bigskip

\begin{center}
\begin{tabular}{|c|c|c|c|}
\hline
死亡 & 視力正常                      & 視覚障害                   & 合計             \\ \hline
0  &                                 &                            &   4161 (96.81\%)        \\ \hline
1  &                                 &                            &  137 (3.19\%)    \\ \hline
合計 & 3971 (100\%)             & 327 (100\%)            & 4298 (100\%)   \\ \hline
\end{tabular}
\end{center}
\bigskip\bigskip\bigskip\bigskip\bigskip

3.1.2 上記の2つの表の数字を使って \(\chi^2\) 統計量を計算せよ (4\%)

\bigskip\bigskip\bigskip\bigskip\bigskip
\bigskip\bigskip\bigskip\bigskip\bigskip
\bigskip\bigskip\bigskip\bigskip\bigskip
\bigskip\bigskip\bigskip\bigskip\bigskip
\bigskip\bigskip\bigskip\bigskip\bigskip

3.1.4 2 \(\times\) 2 の分割表では,自由度は \(\rule{3cm}{0.2mm}\) (2\%)

\newpage

3.1.5
視覚障害と死亡の関係を示すテーブルのデータをもとに,下表を完成せよ:(6\%)

\bigskip\bigskip

\begin{center}
\begin{tabular}{|c|c|c|c|}
\hline
                 & 視力正常   & 視覚障害    & 合計   \\ \hline
リスク (risk)       &  &   & 0.0319 \\ \hline
オッズ (odds)       &  &   & 0.0329 \\ \hline
対数オッズ (log-odds) &  &  & -3.414 \\ \hline
\end{tabular}
\end{center}

\bigskip

視覚障害と死亡の関連を示すオッズ比を算出せよ:(2\%) \[
\mathbf{OR} = 
\]

このオッズ比の対数を取った値 \(\mathbf{log(OR)}\) は:(2\%)

\[
\mathbf{log(OR)} = 
\]

3.2 年齢の影響を考慮する

\begin{center}
\begin{tabular}{|c|c|c|c|c|c|c|c|c|c|c|}
\hline
       & \multicolumn{10}{c|}{視覚障害 (0 = no, 1 = yes)}                                                                                                  \\ \hline
死亡     & 0             & 1          & 0            & 1           & 0            & 1           & 0           & 1           & 0            & 1           \\ \hline
1 = yes & 29            & 2          & 38           & 10          & 15           & 11          & 15          & 17          & 97           & 40          \\ \hline
0 = no & 2301          & 22         & 1271         & 124         & 212          & 69          & 90          & 72          & 3874         & 287         \\ \hline
  n     &           &           &          &          &           &           &           &           &          &          \\ \hline
年齢     & \multicolumn{2}{c|}{15-34} & \multicolumn{2}{c|}{35-54} & \multicolumn{2}{c|}{55-64} & \multicolumn{2}{c|}{65 +} & \multicolumn{2}{c|}{Total} \\ \hline
\end{tabular}
\end{center}

上記のデータをよく見ると,視覚障害のオッズは年齢と共に上昇している
(年齢が15-34歳群の\((2 + 22) / (29 + 2301) = 0.010\)から年齢が65歳以上群の\((17+72)/(15+90) = 0.848\)に上がっている).しかし,年齢の上昇と共に,死亡のオッズも上がる.年齢はここで,\underline{交絡因子 (confounder) }と定義される.

3.2.1 以上のデータと解説をよく理解した上で,下表を完成せよ:(8\%)

\begin{center}
\begin{tabular}{|c|c|c|c|}
\hline
      & \multicolumn{2}{c|}{オッズ}    &        \\ \hline
年齢    & 視力正常              & 視覚障害    & オッズ比   \\ \hline
15-34 & 29/2301 = 0.01260 &   &   \\ \hline
35-54 & 0.02990           &  &  \\ \hline
55-64 & 0.07075           &  &  \\ \hline
65+   & 0.16667           &  &  \\ \hline
\end{tabular}
\end{center}

各年齢層では視覚障害と死亡との関連はどう変化しているか?(2\%)

\newpage

3.2.2.3 単変量ロジスティック回帰モデルで評価した粗オッズ比 (crude odds
ratio) と比べ,年齢調整オッズ比はどう変わったかを説明せよ.(10\%)

\bigskip\bigskip\bigskip\bigskip\bigskip
\bigskip\bigskip\bigskip\bigskip\bigskip
\bigskip\bigskip\bigskip\bigskip\bigskip

\hypertarget{-10}{%
\section{問題4:生存分析 (10\%)}\label{-10}}

\begin{itemize}
\tightlist
\item
  単変量ハザード比,及び信頼区間の意味を説明せよ.(5\%)
\end{itemize}

\bigskip\bigskip\bigskip\bigskip\bigskip
\bigskip\bigskip\bigskip\bigskip\bigskip
\bigskip\bigskip\bigskip\bigskip\bigskip

\begin{itemize}
\tightlist
\item
  年齢調整ハザード比,及び信頼区間の意味を説明せよ.(5\%)
\end{itemize}


\end{document}
