\documentclass[11pt,]{problemset}
\usepackage{lmodern}
\usepackage{amssymb,amsmath}
\usepackage{ifxetex,ifluatex}
\usepackage{fixltx2e} % provides \textsubscript
\ifnum 0\ifxetex 1\fi\ifluatex 1\fi=0 % if pdftex
  \usepackage[T1]{fontenc}
  \usepackage[utf8]{inputenc}
\else % if luatex or xelatex
  \ifxetex
    \usepackage{mathspec}
  \else
    \usepackage{fontspec}
  \fi
  \defaultfontfeatures{Ligatures=TeX,Scale=MatchLowercase}
    \setmainfont[]{IPAPMincho}
    \setsansfont[]{IPAPGothic}
\fi
% use upquote if available, for straight quotes in verbatim environments
\IfFileExists{upquote.sty}{\usepackage{upquote}}{}
% use microtype if available
\IfFileExists{microtype.sty}{%
\usepackage{microtype}
\UseMicrotypeSet[protrusion]{basicmath} % disable protrusion for tt fonts
}{}
\usepackage{hyperref}
\hypersetup{unicode=true,
            pdfborder={0 0 0},
            breaklinks=true}
\urlstyle{same}  % don't use monospace font for urls
\usepackage{color}
\usepackage{fancyvrb}
\newcommand{\VerbBar}{|}
\newcommand{\VERB}{\Verb[commandchars=\\\{\}]}
\DefineVerbatimEnvironment{Highlighting}{Verbatim}{commandchars=\\\{\}}
% Add ',fontsize=\small' for more characters per line
\usepackage{framed}
\definecolor{shadecolor}{RGB}{248,248,248}
\newenvironment{Shaded}{\begin{snugshade}}{\end{snugshade}}
\newcommand{\AlertTok}[1]{\textcolor[rgb]{0.94,0.16,0.16}{#1}}
\newcommand{\AnnotationTok}[1]{\textcolor[rgb]{0.56,0.35,0.01}{\textbf{\textit{#1}}}}
\newcommand{\AttributeTok}[1]{\textcolor[rgb]{0.77,0.63,0.00}{#1}}
\newcommand{\BaseNTok}[1]{\textcolor[rgb]{0.00,0.00,0.81}{#1}}
\newcommand{\BuiltInTok}[1]{#1}
\newcommand{\CharTok}[1]{\textcolor[rgb]{0.31,0.60,0.02}{#1}}
\newcommand{\CommentTok}[1]{\textcolor[rgb]{0.56,0.35,0.01}{\textit{#1}}}
\newcommand{\CommentVarTok}[1]{\textcolor[rgb]{0.56,0.35,0.01}{\textbf{\textit{#1}}}}
\newcommand{\ConstantTok}[1]{\textcolor[rgb]{0.00,0.00,0.00}{#1}}
\newcommand{\ControlFlowTok}[1]{\textcolor[rgb]{0.13,0.29,0.53}{\textbf{#1}}}
\newcommand{\DataTypeTok}[1]{\textcolor[rgb]{0.13,0.29,0.53}{#1}}
\newcommand{\DecValTok}[1]{\textcolor[rgb]{0.00,0.00,0.81}{#1}}
\newcommand{\DocumentationTok}[1]{\textcolor[rgb]{0.56,0.35,0.01}{\textbf{\textit{#1}}}}
\newcommand{\ErrorTok}[1]{\textcolor[rgb]{0.64,0.00,0.00}{\textbf{#1}}}
\newcommand{\ExtensionTok}[1]{#1}
\newcommand{\FloatTok}[1]{\textcolor[rgb]{0.00,0.00,0.81}{#1}}
\newcommand{\FunctionTok}[1]{\textcolor[rgb]{0.00,0.00,0.00}{#1}}
\newcommand{\ImportTok}[1]{#1}
\newcommand{\InformationTok}[1]{\textcolor[rgb]{0.56,0.35,0.01}{\textbf{\textit{#1}}}}
\newcommand{\KeywordTok}[1]{\textcolor[rgb]{0.13,0.29,0.53}{\textbf{#1}}}
\newcommand{\NormalTok}[1]{#1}
\newcommand{\OperatorTok}[1]{\textcolor[rgb]{0.81,0.36,0.00}{\textbf{#1}}}
\newcommand{\OtherTok}[1]{\textcolor[rgb]{0.56,0.35,0.01}{#1}}
\newcommand{\PreprocessorTok}[1]{\textcolor[rgb]{0.56,0.35,0.01}{\textit{#1}}}
\newcommand{\RegionMarkerTok}[1]{#1}
\newcommand{\SpecialCharTok}[1]{\textcolor[rgb]{0.00,0.00,0.00}{#1}}
\newcommand{\SpecialStringTok}[1]{\textcolor[rgb]{0.31,0.60,0.02}{#1}}
\newcommand{\StringTok}[1]{\textcolor[rgb]{0.31,0.60,0.02}{#1}}
\newcommand{\VariableTok}[1]{\textcolor[rgb]{0.00,0.00,0.00}{#1}}
\newcommand{\VerbatimStringTok}[1]{\textcolor[rgb]{0.31,0.60,0.02}{#1}}
\newcommand{\WarningTok}[1]{\textcolor[rgb]{0.56,0.35,0.01}{\textbf{\textit{#1}}}}
\usepackage{graphicx,grffile}
\makeatletter
\def\maxwidth{\ifdim\Gin@nat@width>\linewidth\linewidth\else\Gin@nat@width\fi}
\def\maxheight{\ifdim\Gin@nat@height>\textheight\textheight\else\Gin@nat@height\fi}
\makeatother
% Scale images if necessary, so that they will not overflow the page
% margins by default, and it is still possible to overwrite the defaults
% using explicit options in \includegraphics[width, height, ...]{}
\setkeys{Gin}{width=\maxwidth,height=\maxheight,keepaspectratio}
\IfFileExists{parskip.sty}{%
\usepackage{parskip}
}{% else
\setlength{\parindent}{0pt}
\setlength{\parskip}{6pt plus 2pt minus 1pt}
}
\setlength{\emergencystretch}{3em}  % prevent overfull lines
\providecommand{\tightlist}{%
  \setlength{\itemsep}{0pt}\setlength{\parskip}{0pt}}
\setcounter{secnumdepth}{5}
% Redefines (sub)paragraphs to behave more like sections
\ifx\paragraph\undefined\else
\let\oldparagraph\paragraph
\renewcommand{\paragraph}[1]{\oldparagraph{#1}\mbox{}}
\fi
\ifx\subparagraph\undefined\else
\let\oldsubparagraph\subparagraph
\renewcommand{\subparagraph}[1]{\oldsubparagraph{#1}\mbox{}}
\fi

%%% Use protect on footnotes to avoid problems with footnotes in titles
\let\rmarkdownfootnote\footnote%
\def\footnote{\protect\rmarkdownfootnote}

%%% Change title format to be more compact
\usepackage{titling}

% Create subtitle command for use in maketitle
\providecommand{\subtitle}[1]{
  \posttitle{
    \begin{center}\large#1\end{center}
    }
}

\setlength{\droptitle}{-2em}

  \title{}
    \pretitle{\vspace{\droptitle}}
  \posttitle{}
    \author{}
    \preauthor{}\postauthor{}
    \date{}
    \predate{}\postdate{}
  
\usepackage[ipa]{zxjafont}
\usepackage{sectsty}
\usepackage{tikz}
\usetikzlibrary{trees}
\usepackage{mathspec}
\usepackage{amsmath,amsthm}
\usepackage{amssymb,amsfonts}
\usepackage[all,arc]{xy}
\usepackage{enumerate}
\usepackage{mathrsfs}
\usepackage[margin=1in]{geometry}
\usepackage{thmtools}
\usepackage{verbatim}
\usepackage{xltxtra}
\XeTeXlinebreaklocale ``ja''
\XeTeXlinebreakskip=0pt plus 1pt
\XeTeXlinebreakpenalty=0
\setmainfont[]{IPAPMincho}
\setsansfont[]{IPAPGothic}
\usepackage{afterpage}
\sectionfont{\fontsize{12}{15}\selectfont}
\author{学籍番号:}
\course{疫学演習 2019-6-5 \& 2019-6-12}
\problemset{ }
\problem{ }
\collab{氏名:}

\begin{document}

\section{問題1:両群間計量データの平均値を比較する}

200名の認知症患者を募集し,認識能力テスト(cognitive test,
COG),及び脳萎縮の進行度 (brain atrophy,
脳体積の平均年間減少率,単位は\%)
の検査を全員に行った.COG,及び脳萎縮のデータは大きいほど認知症の進行度がより進んでいる.また,この200名の参加者から採取した血液検体を利用して,ある遺伝子の変異の有無を検査した.このデータは以下の表でまとめた:

\bigskip
\begin{center}
\begin{tabular}{|l|l|l|l|l|}\hline
変数                     & \multicolumn{2}{c|}{遺伝変異あり (n = 50)}     & \multicolumn{2}{c|}{遺伝変異なし (n = 150)}   \\\hline
                       & \vtop{\hbox{\strut 平均値}\hbox{\strut (mean)}}&\vtop{\hbox{\strut  標準偏差}\hbox{\strut (standard deviation) }} & \vtop{\hbox{\strut 平均値}\hbox{\strut (mean)}}&\vtop{\hbox{\strut  標準偏差}\hbox{\strut (standard deviation) }} \\\hline
認識能力テスト,COG            & 69.2       & 9.0                       & 60.2       & 9.0                       \\
脳萎縮度, atrophy, \%/year & 0.67       & 0.21                      & 0.23       & 0.10                     \\\hline
\end{tabular}
\end{center}
\bigskip

\begin{enumerate}
\def\labelenumi{\arabic{enumi}.}
\tightlist
\item
  帰無仮説を「遺伝子変異ありと変異なし両群の間に,COGの平均値は等しい」とする.上記のデータ及び適宜な方法を使って検定せよ.検定の結果を分かりやすく説明せよ.なお,分散が等しいと仮定できる場合,以下の式で両群の共通標準偏差が計算できる:
\end{enumerate}

\begin{equation}
  \label{eq:1}
S = \sqrt{\frac{(n_A - 1)S^2_A + (n_B - 1)S^2_B}{n_A + n_B -2}}
\end{equation}

\begin{itemize}
\tightlist
\item
  \(S_A:\) A群の標準偏差;
\item
  \(n_A:\) A群の人数;
\item
  \(S_B:\) B群の標準偏差;
\item
  \(n_B:\) B群の人数;
\item
  \(S:\) A群及びB群の共通標準偏差;
\item
  \(n_A + n_B -2:\) 分散が等しい時の自由度.
\end{itemize}

また,EZR で t 値,自由度 (degree of freedom)を使って P
値を計算する時,以下のコマンドを利用してください:

\begin{Shaded}
\begin{Highlighting}[]
\DecValTok{2}\OperatorTok{*}\KeywordTok{pt}\NormalTok{(t value, degree of freedom, }\DataTypeTok{lower=}\OtherTok{FALSE}\NormalTok{)}
\end{Highlighting}
\end{Shaded}

\newpage
\vfill

\subsection{答え}

\newpage
\vfill

以下のコードをRスクリプトに入力して,実行をクリックしてください.自分の検定結果とは一致するかを確認してください.

\begin{Shaded}
\begin{Highlighting}[]
\KeywordTok{source}\NormalTok{(}\StringTok{"http://aoki2.si.gunma-u.ac.jp/R/src/my_t_test.R"}\NormalTok{, }\DataTypeTok{encoding=}\StringTok{"euc-jp"}\NormalTok{)}
\KeywordTok{my.t.test}\NormalTok{(}\DecValTok{50}\NormalTok{, }\FloatTok{69.2}\NormalTok{, }\FloatTok{9.0}\OperatorTok{^}\DecValTok{2}\NormalTok{, }\DecValTok{150}\NormalTok{, }\FloatTok{60.2}\NormalTok{, }\FloatTok{9.0}\OperatorTok{^}\DecValTok{2}\NormalTok{, }\DataTypeTok{var.equal=}\OtherTok{TRUE}\NormalTok{)}
\end{Highlighting}
\end{Shaded}

\bigskip\bigskip\bigskip\bigskip\bigskip\bigskip\bigskip

\begin{center}\includegraphics[width=0.9\linewidth]{pic/myttest01cut} \end{center}
\bigskip

\newpage
\vfill

\begin{enumerate}
\def\labelenumi{\arabic{enumi}.}
\setcounter{enumi}{1}
\tightlist
\item
  この患者データから,遺伝子変異ありとなしの群の間に脳萎縮度 (atrophy)
  の比較を 1. と同じ方法で検定してもよいか?どの検定方法を使えば 1.
  と同じ検定方法を使えるかどうかを判断できるを説明せよ.実際にこの検定方法を行ってください.
\end{enumerate}

なお,EZR で F 値,両群の分散,両群それぞれの自由度 (df) を使って P
値を計算する時に,以下のコマンドを利用してください:

\begin{Shaded}
\begin{Highlighting}[]
\DecValTok{2}\OperatorTok{*}\KeywordTok{pf}\NormalTok{(F value, df }\ControlFlowTok{in}\NormalTok{ group }\DecValTok{1}\NormalTok{, df }\ControlFlowTok{in}\NormalTok{ group }\DecValTok{2}\NormalTok{, }\DataTypeTok{lower=}\OtherTok{FALSE}\NormalTok{)}
\end{Highlighting}
\end{Shaded}

\hypertarget{-1}{%
\subsection{答え}\label{-1}}

\bigskip
\newpage
\vfill

以下のコードをRスクリプトに入力して,実行をクリックしてください.自分の検定結果とは一致するかを確認してください.

\begin{Shaded}
\begin{Highlighting}[]
\KeywordTok{source}\NormalTok{(}\StringTok{"http://aoki2.si.gunma-u.ac.jp/R/src/my_var_test.R"}\NormalTok{, }\DataTypeTok{encoding=}\StringTok{"euc-jp"}\NormalTok{)}
\KeywordTok{my.var.test}\NormalTok{(}\DecValTok{50}\NormalTok{, }\FloatTok{0.21}\OperatorTok{^}\DecValTok{2}\NormalTok{, }\DecValTok{150}\NormalTok{, }\FloatTok{0.1}\OperatorTok{^}\DecValTok{2}\NormalTok{)}
\end{Highlighting}
\end{Shaded}

\bigskip\bigskip\bigskip\bigskip\bigskip

\begin{center}\includegraphics[width=0.9\linewidth]{pic/ftest01cut} \end{center}

\newpage
\vfill

\begin{enumerate}
\def\labelenumi{\arabic{enumi}.}
\setcounter{enumi}{2}
\tightlist
\item
  2.の結果を踏まえて,帰無仮設「両群の脳萎縮度の平均値が等しい」を検定せよ.なお,両群の分散が等しいという前提が満たされていない時に,自由度(df)の計算式は以下となる:
\end{enumerate}

\begin{equation}
\label{eq:2}
\mathbf{df} = \frac{(S^2_A/n_A + S^2_B/n_B)^2}{(S_A^2/n_A)^2/(n_A-1)+(S_B^2/n_B)^2/(n_B-1)}
\end{equation}

また,EZR で t 値,自由度 (df)を使って P
値を計算する時,以下のコマンドを利用してください:

\begin{Shaded}
\begin{Highlighting}[]
\DecValTok{2}\OperatorTok{*}\KeywordTok{pt}\NormalTok{(t value, df, }\DataTypeTok{lower=}\OtherTok{FALSE}\NormalTok{)}
\end{Highlighting}
\end{Shaded}

\hypertarget{-2}{%
\subsection{答え}\label{-2}}

\newpage
\vfill

以下のコードをRスクリプトに入力して,実行をクリックしてください.自分の検定結果とは一致するかを確認してください.

\begin{Shaded}
\begin{Highlighting}[]
\KeywordTok{source}\NormalTok{(}\StringTok{"http://aoki2.si.gunma-u.ac.jp/R/src/my_t_test.R"}\NormalTok{, }\DataTypeTok{encoding=}\StringTok{"euc-jp"}\NormalTok{)}
\KeywordTok{my.t.test}\NormalTok{(}\DecValTok{50}\NormalTok{, }\FloatTok{0.67}\NormalTok{, }\FloatTok{0.21}\OperatorTok{^}\DecValTok{2}\NormalTok{, }\DecValTok{150}\NormalTok{, }\FloatTok{0.23}\NormalTok{, }\FloatTok{0.10}\OperatorTok{^}\DecValTok{2}\NormalTok{, }\DataTypeTok{var.equal=}\OtherTok{FALSE}\NormalTok{)}
\end{Highlighting}
\end{Shaded}

\bigskip\bigskip\bigskip\bigskip\bigskip

\begin{center}\includegraphics[width=0.9\linewidth]{pic/myttest02cut} \end{center}
\bigskip

\newpage
\vfill

\section{問題2:線形回帰モデル}

190名の乳幼児の性別(1 = 男,2 = 女),年齢 (月,
months),体重(kg)のデータを収集した.このデータを用いで,以下の問題を解答したい:

\begin{itemize}
\item
  子供の年齢が一ヶ月の増加によって,体重はどれぐらい増えているか?
\item
  男の子は女の子と比べて,平均的に体重はどれぐらい大きい/小さい?
\end{itemize}

\subsection{データのインポート}

\subsubsection{ステップ 1}

\begin{center}\includegraphics[width=0.5\linewidth]{pic/import00} \end{center}

\hypertarget{-1}{%
\subsubsection{ステップ2}\label{-1}}

\begin{center}\includegraphics[width=0.5\linewidth]{pic/import02} \end{center}

\hypertarget{-2}{%
\subsubsection{ステップ3}\label{-2}}

\begin{center}\includegraphics[width=0.7\linewidth]{pic/import03} \end{center}

\hypertarget{-3}{%
\subsubsection{ステップ4}\label{-3}}

\begin{center}\includegraphics[width=0.5\linewidth]{pic/import04} \end{center}

\hypertarget{-4}{%
\subsubsection{ステップ5}\label{-4}}

\begin{center}\includegraphics[width=0.15\linewidth]{pic/import05} \end{center}

\subsection{体重と年齢の散布図,性別により体重の箱ひげ図}

\begin{center}\includegraphics[width=0.85\linewidth]{pic/scatter00} \end{center}

上記左のグラフを描くため,以下のコードをRスクリプトに入力して,実行をクリックしてください.

\begin{Shaded}
\begin{Highlighting}[]
\KeywordTok{plot}\NormalTok{(Dataset}\OperatorTok{$}\NormalTok{age,Dataset}\OperatorTok{$}\NormalTok{wt, }
     \DataTypeTok{xlab =} \StringTok{"年齢 (months)"}\NormalTok{, }\DataTypeTok{ylab =} \StringTok{"体重 (kg)"}\NormalTok{,}
     \DataTypeTok{main =} \StringTok{"体重と年齢の散布図"}\NormalTok{,}
     \DataTypeTok{ylim =} \KeywordTok{c}\NormalTok{(}\DecValTok{3}\NormalTok{, }\DecValTok{20}\NormalTok{), }\DataTypeTok{pch=}\KeywordTok{c}\NormalTok{(}\DecValTok{2}\NormalTok{))}
\end{Highlighting}
\end{Shaded}

\bigskip

\begin{center}\includegraphics[width=0.8\linewidth]{pic/scatter01} \end{center}

\newpage
\vfill

性別により体重の箱ひげ図を描くため,以下のコードをRスクリプトに入力して,実行をクリックしてください.

\begin{Shaded}
\begin{Highlighting}[]
\KeywordTok{boxplot}\NormalTok{(Dataset}\OperatorTok{$}\NormalTok{wt }\OperatorTok{~}\StringTok{ }\NormalTok{Dataset}\OperatorTok{$}\NormalTok{sex, }\DataTypeTok{ylim =} \KeywordTok{c}\NormalTok{(}\DecValTok{0}\NormalTok{, }\DecValTok{18}\NormalTok{),}
        \DataTypeTok{main =} \StringTok{"体重と年齢"}\NormalTok{, }\DataTypeTok{xlab =} \StringTok{"性別"}\NormalTok{, }\DataTypeTok{ylab =} \StringTok{"体重 (kg)"}\NormalTok{,}
        \DataTypeTok{col =} \StringTok{"green"}\NormalTok{, }\DataTypeTok{names =} \KeywordTok{c}\NormalTok{(}\StringTok{"男性"}\NormalTok{, }\StringTok{"女性"}\NormalTok{))}
\end{Highlighting}
\end{Shaded}

\begin{center}\includegraphics[width=0.8\linewidth]{pic/box01} \end{center}

\hypertarget{ezr-mean--cor1-2-}{%
\subsection{\texorpdfstring{年齢,体重それぞれの平均値,分散を求めよ;また,年齢と体重の相関係数を算出せよ.なお,EZRで計量データの平均値を計算するには,コマンド
\texttt{mean(変数名)} を使う;共分散を計算したい時に,コマンド
\texttt{cor(変数1,\ 変数2)}
を利用する.}{年齢,体重それぞれの平均値,分散を求めよ;また,年齢と体重の相関係数を算出せよ.なお,EZRで計量データの平均値を計算するには,コマンド mean(変数名) を使う;共分散を計算したい時に,コマンド cor(変数1, 変数2) を利用する.}}\label{ezr-mean--cor1-2-}}

以下のコードをRスクリプトに入力して,実行をクリックしてください.(結果を下の余白に記入すること)

\begin{Shaded}
\begin{Highlighting}[]
\CommentTok{# 年齢の平均値}
\KeywordTok{mean}\NormalTok{(Dataset}\OperatorTok{$}\NormalTok{age)}
\CommentTok{# 年齢の分散}
\KeywordTok{var}\NormalTok{(Dataset}\OperatorTok{$}\NormalTok{age)}
\CommentTok{# 体重の平均値}
\KeywordTok{mean}\NormalTok{(Dataset}\OperatorTok{$}\NormalTok{wt)}
\CommentTok{# 体重の分散}
\KeywordTok{var}\NormalTok{(Dataset}\OperatorTok{$}\NormalTok{wt)}
\CommentTok{# 体重と年齢の共分散 covariance}
\KeywordTok{cov}\NormalTok{(Dataset}\OperatorTok{$}\NormalTok{wt, Dataset}\OperatorTok{$}\NormalTok{age)}
\end{Highlighting}
\end{Shaded}

\newpage
\vfill

\hypertarget{barx--x-}{%
\subsection{\texorpdfstring{年齢を説明変数,体重を目的変数とする場合,年齢の傾き(回帰係数),と切片を求めよ.なお,分散と共分散の定義を以下とする,\(\bar{X}\)
は \(X\)
の平均値を示す:}{年齢を説明変数,体重を目的変数とする場合,年齢の傾き(回帰係数),と切片を求めよ.なお,分散と共分散の定義を以下とする,\textbackslash{}bar\{X\} は X の平均値を示す:}}\label{barx--x-}}

\begin{itemize}
\tightlist
\item
  分散 variance:
\end{itemize}

\[
\begin{aligned}
\mathbf{Var}(X) & = \frac{(X_1-\bar{X})^2+(X_2-\bar{X})^2+\cdots+(X_n-\bar{X})^2}{n - 1} \\
                & = \frac{\sum_{i=1}^n(X_i-\bar{X})^2}{n -1}
\end{aligned}
\]

\begin{itemize}
\tightlist
\item
  共分散 covariance:
\end{itemize}

\[
\begin{aligned}
\mathbf{Cov}(X, Y) & = \frac{(X_1 - \bar{X})(Y_1-\bar{Y}) + (X_2 - \bar{X})(Y_2-\bar{Y}) + \cdots + (X_n - \bar{X})(Y_n-\bar{Y})}{n - 1} \\
                   & = \frac{\sum_{i = 1}^n(X_i - \bar{X})(Y_i-\bar{Y})}{n - 1}
\end{aligned}
\]

以下のコードをRスクリプトに入力して,実行をクリックしてください.(結果を下の余白に記入すること)

\begin{Shaded}
\begin{Highlighting}[]
\CommentTok{# 傾き (slope)}
\NormalTok{beta <-}\StringTok{ }\KeywordTok{cov}\NormalTok{(Dataset}\OperatorTok{$}\NormalTok{wt, Dataset}\OperatorTok{$}\NormalTok{age) }\OperatorTok{/}\StringTok{ }\KeywordTok{var}\NormalTok{(Dataset}\OperatorTok{$}\NormalTok{age)}
\NormalTok{beta}
\CommentTok{# 切片 (intercept)}
\NormalTok{alpha <-}\StringTok{ }\KeywordTok{mean}\NormalTok{(Dataset}\OperatorTok{$}\NormalTok{wt) }\OperatorTok{-}\StringTok{ }\KeywordTok{mean}\NormalTok{(Dataset}\OperatorTok{$}\NormalTok{age)}\OperatorTok{*}\NormalTok{beta}
\NormalTok{alpha}
\end{Highlighting}
\end{Shaded}

\begin{center}\includegraphics[width=0.8\linewidth]{pic/betaalpha} \end{center}

\hypertarget{ezr}{%
\subsection{実際にEZRで線形モデルを作って見よう:}\label{ezr}}

\hypertarget{-5}{%
\subsubsection{ステップ1}\label{-5}}

\begin{center}\includegraphics[width=0.8\linewidth]{pic/lm00} \end{center}

\hypertarget{-6}{%
\subsubsection{ステップ2}\label{-6}}

\begin{center}\includegraphics[width=0.6\linewidth]{pic/lm01} \end{center}

\hypertarget{-7}{%
\subsubsection{ステップ3}\label{-7}}

\begin{center}\includegraphics[width=0.8\linewidth]{pic/lm02} \end{center}

自分の計算結果とは一致するかを確認してください.

\subsection{今まで計算した傾きと切片の数字を用いて,年齢と体重の関係を線形と考える場合の計算式を記入せよ.傾きと切片の計算結果の意味をそれぞれ記述せよ.}

\hypertarget{-3}{%
\subsubsection{答え}\label{-3}}

\newpage
\vfill

\subsection{性別を説明変数に入れたモデルを作る}

\hypertarget{-factor-}{%
\subsubsection{性別変数を因子 (factor) に変換する}\label{-factor-}}

\hypertarget{-8}{%
\paragraph{ステップ1}\label{-8}}

\begin{center}\includegraphics[width=0.8\linewidth]{pic/sexfactor00} \end{center}

\hypertarget{-9}{%
\paragraph{ステップ2}\label{-9}}

\begin{center}\includegraphics[width=0.4\linewidth]{pic/sexfactor01} \end{center}

\hypertarget{-10}{%
\paragraph{ステップ3}\label{-10}}

\begin{center}\includegraphics[width=0.15\linewidth]{pic/sexfactor02} \end{center}

\paragraph{ステップ4--水準名に男性,女性を入力する}

\begin{center}\includegraphics[width=0.15\linewidth]{pic/sexfactor03} \end{center}

\subsubsection{重回帰線形モデルを作る}

\hypertarget{-11}{%
\paragraph{ステップ1}\label{-11}}

\begin{center}\includegraphics[width=0.8\linewidth]{pic/lm00} \end{center}
\bigskip
\bigskip
\bigskip

\hypertarget{-control-}{%
\paragraph{ステップ2---複数の説明変数を選択する時に control
キーを押しながらマウスで変数名をクリックする}\label{-control-}}

\begin{center}\includegraphics[width=0.4\linewidth]{pic/lm04} \end{center}

\newpage
\vfill

\subsubsection{重回帰線形モデルの結果を確認する}

\begin{center}\includegraphics[width=0.8\linewidth]{pic/lm05} \end{center}

\subsection{重回帰線形モデルの計算結果を用いて,体重の平均値を年齢と性別の線形モデルで表示せよ.各回帰係数の意味を説明せよ.}

\hypertarget{-4}{%
\subsection{答え}\label{-4}}

\newpage
\vfill

\hypertarget{34}{%
\subsection{上記の重回帰線形モデルを用いて,年齢が34ヶ月の女の子の体重の予測値を計算せよ.}\label{34}}

\hypertarget{-5}{%
\subsection{答え}\label{-5}}

\bigskip\bigskip\bigskip\bigskip\bigskip
\bigskip\bigskip\bigskip\bigskip\bigskip
\bigskip\bigskip\bigskip\bigskip\bigskip

\subsection{男女別の年齢と体重の散布図を描く}

\begin{center}\includegraphics[width=0.5\linewidth]{pic/scatter02} \end{center}

\hypertarget{-12}{%
\subsubsection{ステップ1}\label{-12}}

\begin{center}\includegraphics[width=0.5\linewidth]{pic/scatter03} \end{center}

\hypertarget{-13}{%
\paragraph{ステップ2}\label{-13}}

\begin{center}\includegraphics[width=0.4\linewidth]{pic/scatter04} \end{center}

\hypertarget{-14}{%
\paragraph{ステップ3}\label{-14}}

\begin{center}\includegraphics[width=0.15\linewidth]{pic/scatter05} \end{center}

\hypertarget{-15}{%
\paragraph{ステップ4}\label{-15}}

\begin{center}\includegraphics[width=0.4\linewidth]{pic/scatter06} \end{center}

\hypertarget{-16}{%
\paragraph{ステップ5}\label{-16}}

\begin{center}\includegraphics[width=0.15\linewidth]{pic/scatter07} \end{center}

\newpage
\vfill

\section{参考図書:}

1.「Rによる保健医療データ解析演習」,中澤 港,(\url{http://minato.sip21c.org/msb/medstatbookx.pdf})
2.「みんなの医療統計 12日間で基礎理論とEZRを完全マスター!」,新谷 歩.


\end{document}
